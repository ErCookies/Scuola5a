\documentclass[a4paper,12pt]{article}
\usepackage[utf8]{inputenc}
\usepackage{geometry}
\geometry{margin=2.5cm}
\usepackage{hyperref}
\usepackage{amsmath}
\title{Relazione tecnica – Amazon Vobarnese}
\author{Cucchi Francesco}
\date{\today}

\begin{document}
\maketitle

\section{Introduzione}
Questo documento descrive la struttura e il funzionamento del sistema Amazon Vobarnese, rappresentato tramite uno schema UML che individua attori e casi d'uso. L’obiettivo è descrivere la struttura del sistema, le sue funzioni e l’integrazione con attori esterni.

\section{Architettura del sistema}
L’architettura mostra una piattaforma e-commerce organizzata in moduli, che gestiscono diverse aree funzionali:
\begin{itemize}
  \item Area pubblica (registrazione e visualizzazione catalogo)
  \item Area autenticata (login, acquisto prodotti, gestione account, richiesta rimborso, assistenza)
  \item Moduli amministrativi (gestione catalogo, gestione personale)
  \item Integrazioni con attori esterni (banca)
\end{itemize}
Ogni componente comunica tramite flussi diretti, rappresentati da linee di relazione che specificano interazioni, input e output dei processi.

\section{Componenti e interfacce}
Componenti individuati:
\begin{itemize}
  \item Registrazione: gestione nuova utenza.
  \item Visualizzazione catalogo: accesso all’offerta prodotti.
  \item Login: autenticazione cliente registrato.
  \item Acquisto prodotto: flusso di acquisto, integrazione con sistema bancario.
  \item Gestione account: gestione dati del profilo cliente.
  \item Rimborso prodotto: flusso di richiesta e autorizzazione rimborso, con possibile integrazione con sistema bancario.
  \item Assistenza clienti: rapporto fra utente e chatbot, o fra utente e personale.
  \item Gestione catalogo/dati prodotto: aggiornamento prodotti, accessibile ai dipendenti ed admin.
  \item Gestione personale: area riservata all’admin.
\end{itemize}

\section{Flussi operativi e funzionalità}
Principali flussi:
\begin{itemize}
  \item Cliente non registrato: può solo visualizzare il catalogo e registrarsi.
  \item Cliente registrato: può accedere a servizi aggiuntivi tra cui login, gestione account, richieste assistenza, acquisto e rimborso prodotti.
  \item Dipendente: gestisce catalogo e dati prodotto, oltre a fornire eventuale assistenza.
  \item Admin: gestisce personale, oltre alle mansioni dedicate ai dipendenti comuni.
\end{itemize}
Sequenza esempio: un cliente registrato effettua il login $\rightarrow$ visualizza catalogo $\rightarrow$ seleziona prodotto $\rightarrow$ avvia acquisto $\rightarrow$ inoltra richiesta al sistema bancario $\rightarrow$ riceve conferma/esito.


\section{Conclusioni}
...

\end{document}
